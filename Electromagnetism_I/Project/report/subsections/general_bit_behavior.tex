
%In this section, the numerical analysis of a bit traveling through a lossless transmission line is %established ({\bf Figure CIRCUIT}). To this end, a rough analytical estimation of the simulation is %made

\subsection{General behavior of a bit traveling through a transmission line}
This section aims to predict, and thus validate, the behavior of the numerical simulation of the voltage in a LTL. A rough analytical approach is used, since it gives more insight in the physical nature of the problem.  \\

Rearranging the telegrapher's equations for a LTL yields

\begin{equation}
\frac{\partial^2\hat{v}(z, t)}{\partial z^2} - k^2\frac{\partial^2 \hat{v}(z, t)}{\partial t^2} = 0, \quad k = \frac{1}{c}= \sqrt{LC} = \mathrm{cte}.
\label{tele}
\end{equation}

This means that $\hat{v}$ satisfies the wave equation and thus can be written as a superposition of two voltage waves traveling in opposite directions with constant speed $c$. That is,

\begin{equation}
\hat{v}(z, t) = \underbrace{\hat{v}^{+}(z - ct)}_{\text{forward wave}} + \underbrace{\hat{v}^{-}(z + ct)}_{\text{backward wave}}.
\end{equation}

Realising that the voltage and current are related;

\begin{equation}
\hat{i}(z, t) = \frac{1}{R_c}(\hat{v}^{+}(z, t) - \hat{v}^{-}(z, t)),
\end{equation}

with $R_c = \sqrt{\frac{L_{TL}}{C_{TL}}}$ and $L_{TL}$, $C_{TL}$ p.u.l. inductance resp. capacitance of the TL, a constant that depends on the characteristics of the TL, hence the name characteristic impedance. Although it has the units of resistance ($\Omega$) it is not reactive; there is no energy dissipation over the TL. The characteristic impedance can be interpreted as a scale for (1) the current to voltage amplitude when there is only a forward or backward wave present in the TL and (2) the dissipation of energy to the load or generator (see (REF)). Furthermore, the CI is the consequence of the model we provided for the TL; it counts for the p.u.l capacitance and inductance (depend on type of TL). In further sections, $R_c$ is assumed to be finite and constant.
