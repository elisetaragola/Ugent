\documentclass{article}

\title{Quantitative Cell and Tissue Analysis}
\author{Lucas Comyn}

\begin{document}
\maketitle


\section{Block I}
 
\begin{table}[h]
\centering
\begin{tabular}{|l|l|}
\hline
\textbf{Term} & \textbf{Definition} \\ \hline
AFM & Atomic force microscopy \\ \hline
DIC & Differential interference microscopy \\ \hline
STED & Stimulated emission depletion microscopy (doughnut) \\ \hline
TIRF & Total internal reflection interference microscopy \\ \hline
PALM & Photo-activated localisation microscopy \\ \hline
FACS & Fluorescence activated cell sorting (using positive and negative charges for binning) \\ \hline
FRAP & Fluorescence recovery after photo bleaching (diffusion of fluorophores) \\ \hline
FCS & Fluorescence correlation spectroscopy \\ \hline
FRET & Fluorescence resonance energy transfer \\ \hline
SEM & Scanning electron microscopy \\ \hline
TEM & Transmission electron microscopy (preparation of sample!) \\ \hline
NRM & Nuclear magnetic resonance \\ \hline
DNA & Deoxyribonucleic acid \\ \hline
RNA & Ribonucleic acid \\ \hline
PCR & Polymerase chain reaction \\ \hline
PMT & Photomultiplier \\ \hline
CCD & Charge coupled device \\ \hline
LED & Light emitting diode \\ \hline
GFP & Green fluorescence proteïn \\ \hline
SE & Secondary electron \\ \hline
BSE & Back-scattered electron \\ \hline
SRRF & Super resolution radial fluctuations (software) \\ \hline
DHM & Digital holographic microscopy (increase z resolution using $\phi$, $\lambda$, and h(height)) \\ \hline
MSI & Mass spectrometry imaging \\ \hline
LC & Liquid chromatography \\ \hline
FCM & Flow cytometry \\ \hline

\end{tabular}
\end{table}



\section{Block II}

\subsection{Abbreviations}

\begin{table}[h]
\begin{tabular}{|l|l|}
\hline
\textbf{Term} & \textbf{Definition} \\ \hline
EV & Extracellular vesicle \\ \hline
RBC & Red blood cell \\ \hline
WBC & White blood cell \\ \hline
LPP & Lipoproteïn particles \\ \hline
SEC & Size exclusion chromatography \\ \hline
UF & Ultrafiltration \\ \hline
ODG & OptiPrep density gradient (Iodixanol) \\ \hline
THP & Tamm - Horsfall proteïn \\ \hline
rEV & Recombinant extra cellular vesicle (it was modified to express a protein more) \\ \hline
TRPS & Tunable resistive pulse sensing \\ \hline
NTA & Nano praticle tracking analysis \\ \hline
AF-MALS & Assymetric flow field flow fractionation coupled with multi-angled light scattering \\ \hline
AF4 & Assymetric flow field flow fractionation \\ \hline
MVE & Multi vesicular endosome  \\ \hline
PBS & phosphate buffered saline \\ \hline
ECM & Extracellullar matrix \\ \hline
BM & Basement membrane \\ \hline
 &  \\ \hline
 &  \\ \hline
      

\end{tabular}
\end{table}

\subsection{Technologies}

\begin{itemize}
    
    \item TRPS uses a tunable voltage, pore size and pressure to filter the particles in a more exact way adn detect their size and concentration. (-) Clotting of the pores is possible. (+) Measures the $\zeta$ potential at once. Which is related to the electrophoretic mobility.
    
    \item ExoView: technology that uses an antibody coated chip to capture EV's on the plate. After that stained antibodies detect the used antibody on these EV's. We could also use a mild detergent (SDS) to visualise the RNA molecules that are in the EV.
    (+) No complicated separation needed assessment is directly possible of the bio fluid. Immediately biological markers... (-) less quantitative (cost).
    
    \item Nano-FCM: low pressure sheath fluid and a reduced flow rate
    (+) estimation on the size, concentration also biological information. Multiple labels can be visualised at once. (-) The fluorescent technology needs to be applied in a fluid and so will also need to be removed again so there is no background signal...
    
    \item AF4: First the smaller and then the larger particles will elude.
    here the labels wont need to be removed because these are very small particles and will so elude the first. (why do small particles elude first? because these have a higher diffusion coefficient and will be able to get higher in the downwards applied stream and so be in the more center of the horizontal flow that is larger in the middle (parabolic pattern)) (+) The fluorescent particles dont need to be removed since these will elude first.
    
    \item EXODUS: Isolation technique. Exosome detection via ultra fast isolation system. Based on negative pressure oscillation and membrane vibration.
    
    \item Anion exchange chromatography: Uses the negative surface charge of EV's to select them (determined by the $\zeta$ potential)
    
    \item ELISA bead-base flow cytometry:
    ELISA or Enzyme-Linked Immunosorbent Assay is an immuno assay technique utilized to detect diseases. The principle of ELISA is antigen-antibody interaction.
    
    \item SDS-Page used to determine protein amount (see later)
    
    \item
\end{itemize}

\subsection{Cell culturing}

37°C + 5\% CO2
cadherins (cell-cell adhesion)(Na2+)
integrins( cell-ECM adhesion)
3 steps in sub culturing:
\begin{itemize}
    \item cell collection
    \item cell counting
    \item cell seeding
\end{itemize}
PH is maintained using a buffer system. Mycoplasms contamination is not visible

But also limitations a cell culture really doesn't simulate real body environment. E.g.: 2d situations (which are stiff resistance to deformation etc), most of the time only 1 cell type at the time. cell lines have been cultured for multiple generations resulting into a drift of their characteristics.
But also limitations due to the culture conditions. see slide 33!!!
\end{document}


