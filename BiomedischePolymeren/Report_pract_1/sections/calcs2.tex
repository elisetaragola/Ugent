\subsection{ATRP}

    The theoretical molar mass of PMMA is the product of the degree of polymerisation
    with the molar mass of the monomer it consists from, i.e. MMA.
    
    $$\mathtext[PMMA]{MW} = \mathtext[PMMA]{DP} \cdot \mathtext[MMA]{MW} = 50 \cdot 190.65\MW =5006,5\MW$$

    From the formula of degree of polymerisation (there is more than one formula)

    \begin{equation}
        \mathtext[PMMA]{DP} = \frac{\text{\# repeat unit}}{\# polymers},
        \label{eq:DPpmma}
    \end{equation}

    one can derive the amount of initiator needed. Since the number of polymers depends
    on the number of initial radicals present which depends on the number of initiator (I) added
    and the amout of repeat unit is equal to the amout of monomer present one can rewrite the above formula as 

    \begin{equation}
        \# initiator = \frac{\text{\# monomer}}{\mathtext[PMMA]{DP}}.
        \label{eq:numbI}
    \end{equation}

    The amout of monomer present in moles can be derived as

    $$\mathtext[MMA]{m} = 0.943\density\cdot 10\ml = 9.43\g,$$
    $$\mathtext[MMA]{n} = \frac{9.43\g}{100.13\MW} = 9.42\cdot10^{-2}\mole.$$

    Converting formula \ref{eq:numbI} to moles yields

    $$\mathtext[I]{n} = \frac{9.42\cdot10^{-2}\mole}{50}=1.88\cdot10^{-3}\mole.$$

    Multiplying this result with the molecular weight of the initiator gives the
    the amount of initiator needed in grams

    $$\mathtext[I]{m} = 1.88\cdot10^{-3}\mole \cdot 190.65\MW = 3.59\cdot10^{-1}\g.$$

    Now that the amount of initiator is know one can calculate the amount of Cu(I)Br and 
    2,2'-bipyridine [bpy] needed from the know molar ratio [Cu(I)Br][bpy][I] = 1:3:4.

    $$\left\{\begin{matrix}
        \mathtext[Cu(i)Br]{n} = \frac{1}{4}\mathtext[I]{n} = 0.47\cdot10^{-3}\mole,\\
        \mathtext[bpy]{n} = \frac{3}{4}\mathtext[I]{n} = 1.41\cdot10^{-3}\mole.\\
    \end{matrix}\right.$$

    Converting to grams gives

    $$\left\{\begin{matrix}
        \mathtext[Cu(i)Br]{m} = 0.47\mole \cdot 143.45\MW = 67.4 \cdot10^{-3}\g,\\
        \mathtext[bpy]{m} = 1.41\mole \cdot 156.18\MW = 220.2 \cdot10^{-3}\g.\\
    \end{matrix}\right.$$

    At last, the amout of solvent (ethyl acetate) which is present is equimolaire 
    with the amount of MMA present, hence its volume can be calculated as

    $$\mathtext[solvent]{m} = \mathtext[MMA]{n}\cdot 88.11\MW = 8.30\g,$$
    $$\mathtext[solvent]{V} = \frac{\mathtext[solvent]{m}}{0.902\density} = 9.20\ml.$$