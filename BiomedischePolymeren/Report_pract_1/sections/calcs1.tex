\subsection{Convential radical polymerisation}

    The amount of reactive units of MMA reflects how much initiator (irgacure 651) is needed.
    The reactive units of MMA is given as $x$ mole\% from the total monomers present. So the first step
    is to calculate the total amount of MMA present.
    $$\mathtext[MMA]{m} = 0.943\density\cdot 4 \ml = 3.722 g$$
    $$\mathtext[MMA]{n} = \frac{\mathtext[MMA]{m}}{100.13\MW} = 3.77 \cdot 10^{-2} \mole$$
    The amount of initiator needed for $x$ mole\% is then given by

    \begin{equation}
        \mathtext[Irg651]{n}(x) = 3.77\cdot10^{-4}\cdot x \;\mole,
        \label{eq:molepercent}
    \end{equation}

    in moles and the expresions in units gram turns into

    \begin{equation}
        \mathtext[Irg651]{m}(x) = 256.3\MW\cdot3.77\cdot10^{-4}\cdot x \;\mole = 9.66\cdot10^{-2}\cdot x \;\g.
        \label{eq:molepercent_grams}
    \end{equation}

    The degree of polymerisation and the theoretical molar mass can both be expressed in function of $x$. Only the theoretical
    case will be considered where every radical will initiate a polymer chain, hence the amount of polymers present will be equal 
    to the amout of initiator at the beginning. Furthermore, the amout of repeat unit is equal to the amout of monomers at the start
    of the reaction.

    \begin{equation}
        \mathtext[PMMA]{DP}(x) = \frac{\# \text{repeat unit}}{\# \text{polymers}} = \frac{\mathtext[MMA]{n}}{\mathtext[Irg651]{n}(x)} = \frac{100}{x} \\
        \label{eq:DPx}
    \end{equation}

    \begin{equation}
        \mathtext[PMMA]{MW}(x) = DP\cdot 100.13\MW = \frac{10013}{x} \MW
        \label{eq:MWx}
    \end{equation}

    Three different values for $x$ where considered. In table \ref{tab:calcs1} the results for these three 
    different values can be found.

    \begin{table}[H]
        \centering
        \begin{tabular}{c|c|c|c}
            & 2 mole \% & 3 mole \% & 4 mole \%\\ \hline
            $\mathtext[Irg651]{n}$ ($10^{-4}$ mol) & 7.54 & 11.31 & 15.08 \\
            $\mathtext[Irg651]{m}$ (mg) & 193.2 & 289.8 & 386.4 \\
            $\mathtext[PMMA]{DP}$ & 50 & 33.3 $\approx$ 30 & 25\\
            $\mathtext[PMMA]{MW}$ ($\MW$) & 5006.5 & 3337.7 & 2503.3
        \end{tabular}
        \caption{results of Calculations}
        \label{tab:calcs1}
    \end{table}